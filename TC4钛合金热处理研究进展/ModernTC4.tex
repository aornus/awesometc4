\documentclass[
class = book,
zihao = -4,
font = noto,
paper = a4paper,
openany
]{easybook}

\usepackage{xju}
\newcommand{\ti}{Ti6Al4V}

\begin{document}
	\hypersetup{pdftitle=Ti-6Al-4V 钛合金热处理工艺的研究现状及进展,pdfauthor=田欣洋,pdfsubject={材料科学,金属学,钛合金},pdfkeywords={热处理,固溶,时效,组织},pdfstartview=FitB}
	\maketitle
	\frontmatter*[roman]

	\begin{abstract}
		\ti 合金又名TC4合金,拥有较好的塑韧性、耐热性、成形性、耐蚀性等,
		%	其使用量已占钛合金使用总量的75$ \% $~85$ \% $,也是大多数高强钛合金的基础,被誉为钛合金中的“王牌合金”,
		在机械、军事、航空航天等领域获得了极为广泛的应用。但TC4合金仍存在硬度较低、摩擦磨损系数高、耐磨性能差、较低的塑韧性和力学性能上的各向异性等缺点,制约了其进一步的应用。本文旨在调研固溶时效处理Ti6Al4V合金强度的影响,并分析了不同固溶时效工艺参数下处理Ti6Al4V合金的力学性能,确定了最佳的固溶温度、时效温度、失效时间等参数,为工程应用提供了有价值的参考。
%		\begin{enumerate}
%			\item (\text{\color{blue}从热处理制度})在950℃进行固溶、550℃进行时效处理时可以得到合金最佳的力学性能。
%			\item (\text{\color{blue}从微观组织})冷却速率越高,得到组织所含$ \beta  $相含量越多,综合性能越好。
%			\item (\text{\color{blue}从转变机理})时效时间越久,亚稳定$ \beta $相分解的就越充分,得到的组织性能更好。
%		\end{enumerate}
\\

		\keywords{Ti-6Al-4V 钛合金;热处理;显微组织;力学性能;现状}
	\end{abstract}

	\tableofcontents
	\mainmatter*

	\pagestyle{Xju}


	\backmatter
	\listoffigures
	\listoftables
	\clearpage
	\phantomsection
	\addcontentsline{toc}{chapter}{参考文献}
	\bibliography{modern}


\end{document}


