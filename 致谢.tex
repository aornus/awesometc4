\chapter{致谢}
%致谢:对于毕业论文(设计)的指导教师,对毕业论文(设计)提过有益的建议或给予过帮助的同学、同事与集体,都应在论文的结尾部分书面致谢,言辞应恳切、实事求是。应注明受何种基金支持(没有可不写)。

首先,我要感谢我的毕设导师杨老师。在杨老师的指导下,我深入学习了针对材料成型及控制工程领域的相关理论和技术知识。在热处理实验和金相制备等实验中,杨老师不仅详细讲解了实验操作步骤,还给予了许多技巧性的指导,提高了我的专业能力与实验水平。在论文撰写的过程中,杨老师认真审核我的论文,并提出了许多宝贵的意见和建议,让我的论文更加完善和优秀。此外,在毕业设计中遇到问题时,杨老师总是不厌其烦地为我讲解分析,让我们深入了解问题本质并提出解决方法。在实验过程中,杨老师也亲力亲为,带领我们做实验,直到最后实验数据的分析和解读,为毕设的圆满结束奠定了坚实的基础。在本次毕业设计过程中,我充分感受到杨老师的教学和研究严谨求实的态度,同时也感受到了他对于学生的关心和支持。在此,我再次向杨老师表达我最真挚的感激之情,感谢他的悉心指导和辛苦付出,为我今后的工作和学习打下了坚实的基础。

其次,感谢学长、同学们以及家里人在这段时间中给予我的帮助。在实验中遇到问题时,学长们总是耐心地解答我的疑问,在实验中给予了我许多宝贵的指导和经验分享,使我受益匪浅。同学们在我遇到实验问题时,也积极给予我帮助和支持,共同探究实验的解决方法。我的家人也一直关心我的学业生活,给予我精神上的鼓励和实际上的支持。没有他们的热情和支持,我不可能顺利完成毕业设计。正是因为家人、同学和学长们的鼓励和支持,不管是在繁忙的学习和实践中还是在生活中,我都学会了坚持、拼搏和奋斗,树立了信心,对未来充满了期待。在此,我要再次向这些曾经关心、支持和帮助过我的人表达一份最真挚的感谢!

四年的大学学习生活让我感悟颇深。在这四年里,我不仅学到了专业知识,还培养了自己的综合能力。在课堂上积极思考交流,积极参与社团活动和公益事业,锻炼了自己的组织、领导和沟通协调能力,为未来的工作奠定了坚实的基础。

最后,对未来我充满信心和期待,我相信通过自己的不懈努力和不断的进修学习,一定会有所成就。

非常感谢所有帮助过我和支持我的人,让我能够平稳地度过这三个月的毕业设计和整个大学阶段!