\chapter{结论与展望}
本设计以\ti 合金为研究对象,对合金设计、进行了不同的热处理,通过室温力学性能测试和金相显微镜等分析方法对合金进行了分析,得到了如下结论:

\begin{enumerate}
	\item 固溶处理温度设置与相变点的相对关系对合金力学性能有显著影响。当固溶温度在相变点以下(910℃)时,随着固溶温度的提高,抗拉强度、屈服强度显著增加,当温度达到相变点(950℃)附近时,达到最大;当温度升高到相变点以上时,随着温度升高,抗拉强度大幅度降低,塑性明显下降,室温力学性能测试中可以看到典型脆性断裂拉伸曲线。
	\item 固溶处理的冷却方式影响合金中初生$ \alpha $相的分布组成。当冷却方式为水冷时,$ \beta $相通过非扩散型相变进行相变,得到了$ \alpha^{\prime} $马氏体;当冷却方式为冷却速度稍慢的油冷时,$ \beta $相可以发生较慢的扩散性相变,最终组织中含有相较粗大的初生$ \alpha $相和片层状的亚稳$ \beta $相组织。
\end{enumerate}

虽然本设计得到了一些研究结果,但由于试验设备的精度和试样的数目所限,对于\ti 合金的固溶时效的最佳工艺参数的确定还需要进一步研究:
\begin{enumerate}
	\item $ \beta $相变点的确定需要更准确地测量,结合传统的元素含量法与计算机技术来获得更准确的结果。
	\item 固溶时效过程的相变过程不容易确定,通过实验结果反推相变过程具有一定的局限性,可以通过相场模拟的方法来更深入地探究相变过程。
	\item 固溶失效热处理参数每一个阶段都互相影响,可以通过大量的实验数据,利用神经网络模型、ChatGPT、人工智能等现代计算机技术来验证参数与最终结果的关联性,以获得更准确的关联性解读。
\end{enumerate}