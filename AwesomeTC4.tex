\documentclass[
class = book,
zihao = -4,
font = noto,
paper = a4paper,
openany
]{easybook}
\usepackage{xju}

\begin{document}
	\hypersetup{pdftitle=热处理温度及冷却速度对Ti6Al4V组织和力学性能的影响,pdfauthor=田欣洋,pdfsubject={材料科学,金属学,钛合金},pdfkeywords={热处理,固溶,时效,相图,组织},pdfstartview=FitB}
	\maketitle
	\begin{abstract}
		%		本文为机械设计基础的课程设计说明书文档,以蜗轮蜗杆减速器为例,从原动机的选择开始到主要零部件的设计,一步步系统完备地设计出了减速器,本文采取了\LaTeX 排版,并使用了自动计算的命令,进一步简化了设计的繁琐步骤。
	\end{abstract}
	\tableofcontents
	\pagestyle{Xju}
	\chapter{绪论}
	\section{钛工业的发展历程与国内外现状}
	\subsection{简介}

	钛(Titanium),原子序数为22,最早于1791年由格雷戈尔在英国康沃尔郡发现,是一种银白色的金属,具有密度小、比强度高、耐高温、化学性质性质稳定等明显优于传统金属的特性而备受重视。钛及钛合金常用来制造飞机、火箭等航天机械,一直以来都是航空航天工业的“脊柱”之一,被誉为“太空机械”\cite{XJYS200102014}。与纯钛一同发展起来的钛合金也毫不逊色,钛合金是在纯钛的基础上添加了各种各样的合金元素而形成的合金,凭借其更高的强度、耐蚀性、抗高温性能,得到了广泛的应用,尤其是在机械制造、航空航天、化工、军工等领域,钛合金的占比更大。钛工业的发展水平在一定程度上是衡量一个国家航空航天、汽车工业等发展水平的重要标志\cite{HSJJ202109005}。

	\subsection{钛与钛合金的特点}
	钛合金具有密度小,强度高的显著特点,相较于高强度钢而言,不仅强度相差无几,而且还具有更大的比强度。


		\begin{table}[htbp]
		\centering
		\label{sec:bqd}
		\caption{不同合金比强度比较表}
		\begin{tabular}{ccccc}
		\hline
		\textbf{合金} & \textbf{镁合金} & \textbf{铝合金} & \textbf{高强钢} & \textbf{钛合金} \\
		\hline
		比强度 & 16 & 21 & 23 & 29 \\
		 \hline
		\end{tabular}
	\end{table}

		钛合金的特点如下\cite{1997titanium}:
	\begin{enumerate}
		\item 熔点高,钛的熔点为1668℃,比铁的熔点还高出138℃。加入合金元素后可以获得极佳的热强性。
		\item 弹性模量低,屈服强度高,适合做弹簧材料,高端赛车内部的弹簧大多数都是由钛合金制成,它同时还具有较好的耐磨性。
		\item 表面极易生成致密的氧化层,在氧化性或中性介质中有较强的耐腐蚀能力。
		\item 此外还有无磁性,,形状记忆性等优良特点。
		\item 化学活性高,当钛加热到500℃以上时,氧化膜变得稀松且易脱落,在熔融状态下,极易发生自然。
		\item 此外,某些钛合金还具有储氢、超导、低阻尼性,生物相容性、形状记忆 、 超弹 、高阻尼等特殊功能。
	\end{enumerate}

	由于钛合金具有以上诸多特点,目前已广泛应用于自动化 、能源 、航空航天 、 医	疗卫生 、 汽车和家电等领域。
	\subsection{国外发展}
	钛工业的发展充满曲折。从钛元素的发现(1791)到第一次制得较纯的金属钛(1910)经历了120年的历程。又由实验室第一次获得纯钛(1940)到首次进行工业生产,又花费了近30年的时间。
	钛在自然界中主要以钛矿石的形式存在,如钛铁矿、金红石(TiO2)等,需要进行精炼(refining)才能获得纯金属。起初,钛的提取是通过高温还原法,但这种方法费时费力,成本高昂。直到了二十世纪四十年代,一种利用氯化钛矿与氯气进行反应来制备四氯化钛,然后通过还原反应(比如Na、Mg等)来得到纯钛的精炼工艺方法终于以其低廉的成本、高效的回收率得到了广泛的商业化应用。

	第二次世界大战之后,世界上许多国家都开始意识到钛工业的重要性,钛工业在数年间便迅速发展成为航空、航天、军事等领域的关键材料。1954年,美国成功研发出一种Ti-6Al-4V合金,这种合金在耐热性、强度、塑性、韧性、成形性、可焊性、耐蚀性和生物相容性方面均达到较高水平,使它成为钛工业的主要合金,并占据全部用钛量的50%以上,可以说,许多其他型号钛合金也可以作为Ti-6Al-4V的改良版。

	进入21 世纪以来,钛工业在多个领域遍地开花。
	\begin{itemize}
		\item 	在航空航天领域中,大型客机的研制如火如荼、军机也处于过渡时期,世界航空工业对钛合金的需求也随之迅猛增长。
		\item 在医疗健康领域,由于钛合金生物相容性良好,可以降低人体对植入物的排斥反应和感染风险,它也被广泛用于制造人工关节、牙科种植体和其他医疗设备。
		\item 在汽车制造领域,钛合金的应用主要集中在高档汽车的制造中。钛合金零部件可以减少车辆的自重,从而提高燃油效率和运行性能。同时,钛合金也具有优异的耐腐蚀性能,可以延长汽车零部件的使用寿命。
		\item 在建筑工程领域,钛合金被广泛应用于大型建筑的外墙幕墙、顶棚和立面系统。钛合金具有良好的耐候性和抗腐蚀性能,可以抵御各种恶劣气候条件的侵蚀,并且具有高度的可塑性和装饰性,可以为建筑带来更加优美的外观。

	\end{itemize}

	\subsection{国内发展}
	我国的钛工业发展起源于20世纪50年代,在六七十年代,成为了世界上第四个拥有完整钛工业体系的国家。自21世纪以来我国钛工业进入高速发展阶段,产能与产量已经连续多年占据世界第一的位置,目前海绵钛产量占全球比重已经达到六成,钛加工材产量稳定增长,钛产品消费端需求旺盛,无论是在生产还是在加工领域均保持在世界前列,我国已成为名副其实的世界钛工业大国。2014年,浙江余杭高端钛材的研发投产,标志着中国彻底摆脱了对国外的依赖,填补了中国高端钛材的技术空白。\cite{TGYJ200405004}\\


	目前,我国的钛产品消费正处于上升期,如工业、航空航天、海洋船舶和体育休闲等中高端领域的钛材料的需求量平均增长约20%,而医疗行业受疫情影响,需求有所减少,电力和制盐等行业仍有小幅增长,整体盈利水平也有所改善。\\


	此外,近年来计算机技术的发展也为钛工业带来了新的发展机遇。计算机模拟技术用于优化钛合金的生产工艺,显著提高了产品质量。邵一涛等通过采用BP人工神经网络方法建立TC17钛合金组织与性能的关系模型,克服了传统BP人工神经网络训练高精度而预测低精度的过拟合问题;计算机辅助设计和制造技术也为钛制品的设计和生产带来了更多的可能,李淼泉等人对 TC6 合金叶片在等温锻造过程中初生α晶粒尺寸的演变进行了数值模拟,将有限元法与 Yada 微观组织模型结合起来,并给出了 TC6 合金叶片在等温锻造过程中初生α相的分布和晶粒尺寸的变化。在未来,随着物联网、大数据、人工智能、AIGC等技术的不断发展,钛工业也将迎来更多新的机遇和挑战。\\

	\subsection{应用领域}
	\begin{enumerate}
		\item  航空航天领域:
		\item  汽车领域:
		\item  化工领域:
		\item  医疗器械领域:
		\item  轻工业领域
	\end{enumerate}


	\section{钛合金的分类}
	\section{钛合金组织分析方法}
	\section{小结}

	\chapter{TC4钛合金的热处理实验}
	\section{TC4型钛合金的热处理工艺}
	\section{TC4钛合金的热处理方案设计}
	\section{TC4钛合金的热处理方案实验过程}
	\section{小结}

	\chapter{TC4钛合金的力学性能实验与组织表征}
	\section{TC4钛合金的力学实验过程}
	\section{TC4钛合金的显微组织表征}
	\section{小结}

	\chapter{综合分析}
	\section{基于机器学习的金相组织分析}
	\section{性能与热处理的关系}
	\section{微观机理}
	\section{结论}

	\chapter{后记}
\bibliography{ciiiiiiiiite}
\end{document}


