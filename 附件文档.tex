\chapter{附录1:不同热处理工艺对Ti6Al4V钛合金微观结构和力学性能影响}

🔤α-phase的衬底阶段(α+β)钛的合金。数量、形状和大小α-phase直接确定的财产(α+β)钛的合金。两阶段地区(α+β)步从热处理得到不同保持时间和温度。温度低于相变温度。(α+β)的主要特征规律微结构形状不规则的谷物、连续和不连续α-phase晶界,以及许多小型辅助α-phases。点状的,球状、片状和短杆α-phase存在于颗粒内的[22]。然而,所有的(α+β)步时将转换成β-phase加热温度高于相变温度。颗粒的大小和形状是不一样的。他们是四边形,五角大楼和六边形。🔤
\chapter{附录2:Effect of Different Heat Treatments on Microstructure and Mechanical Properties of Ti6Al4V Titanium Alloy}
The α-phase is the substrate phase of (α+β)-titanium alloy. The number, shape and size of α-phase determine directly the property of (α+β)-titanium alloy. In the two-phase region, (α+β)-phase is gotten from the heat treatment with different holding time and temperatures. The temperature is below the phase transition temperature. The main characteristics of (α+β)-phase microstructure are irregular shape of grains, continuous and discontinuous α-phase on the grain boundary, and many small secondary α-phases. The punctate, spherical, flakiness and short rod α-phase exists in intragranular[22]. However, all (α+β)-phase will be converted into β-phase when the heating temperature is higher than phase transition temperature. The size and shape of grains are not identical. They are quadrilateral, pentagon and hexagon.
%附录编号依次编为附录1,附录2。附录标题各占一行,按一级标题编排。每一个附录一般应另起一页编排,如果有多个较短的附录,也可接排。附录中的图表公式另行编排序号,与正文分开,编号前加“附录1-”字样。
%
% 每位学生须阅读一定的专业外文资料(专著、期刊、学位论文、论文集、报纸文章、报告、标准、专利、教材、网络资料等),通过文献查阅与阅读,进一步提高使用外文的能力,熟悉本专业的几种主要外文书刊,了解毕业论文(设计)课题的国内外信息与动态。
%
% 翻译的原文应该是来源于学校认可的数据库资源,且是与毕业论文(设计)题目密切相关的资料。译文字数3000字符左右,要求译文与原文内容相符。
%
% 译文要求:
% \begin{enumerate}
% 	\item (1)标题 (2)署名 (3)翻译正文 (4)外文著录
% 	\item 附被翻译文字资料原件的复印件
% \end{enumerate}
