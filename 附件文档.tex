\chapter{附录1:不同热处理工艺对Ti6Al4V钛合金微观结构和力学性能影响}
\begin{center}
	作者:Liu Wanying, Lin Yuanhua Chen Yuhai, Shi Taihe, Ambrish Singh\\
\end{center}
\section*{1. 摘要}
本文分析了不同热处理后Ti6Al4V钛合金的微观结构。通过仪器进行了拉伸和冲击试验,并通过金相显微镜和扫描电子显微镜(ESEM)分析了微观结构、冲击断裂特征和机械性能之间的关系。结果表明,Ti6Al4V钛合金的微观结构、力学性能和冲击韧性受到工艺制度和时效处理的影响。

屈服强度和极限抗拉强度得到了显著提高,延伸率先增加后下降。
当Ti6Al4V钛合金在960℃/h+水冷和500℃/4h+空冷的条件下处理时,可以获得良好的综合性能,其屈服强度 $\sigma_{0.2}$ 为1050MPa、抗拉强度 $\sigma_b$为1120MPa、冲击韧性$A_k$为46.22 $J·cm^{-2}$,经过固溶和时效处理的钛合金的微观结构由$\beta$基体和初生$\alpha$相成,片层状的$\beta$相和针状的$\alpha$相结构可以提高合金的综合性能。

\noindent\keywords{热处理; Ti6Al4V 钛合金;微观组织;力学性能}\\

钛及其合金是航空航天工业的理想材料,因为它具有极佳的高温强度。此外,由于其良好的耐腐蚀性,它也是海洋、石油、化工、制药和其他行业的理想材料。

随着现代石油工业的快速发展,石油和天然气钻探对钻探技术提出了更高的要求,尤其是一些特殊的钻井过程。但许多传统的钻井工具无法满足钻井要求,为了满足特殊、复杂的钻井需求,一系列的钻井工具应运而生。其中,钛合金钻杆是新近开发的一个新产品。与传统的钢质钻杆相比,它的结构应力小、韧性好、抗疲劳,耐腐蚀、质量轻。此外,它在高曲率钻孔作业中也具有良好的适应性。

由于Ti6Al4V合金具有高抗拉强度和抗疲劳强度、低弹性模量、低密度、高硬度和良好的耐腐蚀性,它经常被用作钻杆。Weatherford 公司的Grant Prideco子公司和RTI国际大托拉斯在德州的子公司TI能源系统已经开发出了钛合金钻杆,它不仅具有钢质钻杆的强度,而且是一种柔性轻质的合成材料,具有抗腐蚀和耐久性。但是,Ti6Al4V合金的韧性较差,这限制了它在油田的广泛应用和推广。

组织决定性能。由于钛合金的微观结构不能被各种热机械处理所改变或控制,只有通过热处理来改善其微观结构和机械性能。一些研究表明,快速热处理可以使铸态($\alpha$+$\beta$)钛合金的晶体结构和机械性能得到明显改善。另一篇论文表明:具有粗大的初生晶粒$\alpha$相的Ti6Al4V合金在预制条件下,需要更长的热处理时间才能细化结构,并适度地改善机械性能。在一些关于原生$\alpha$相的晶粒尺寸的研究中发现,$\alpha$+$\beta$两相区会随着变形温度的升高而变化,同时,初生$\alpha$相的体积分数会逐渐下降。热处理在影响钛合金的微观结构和改善综合性能方面具有非常重要的作用。本文的目的是通过探索不同的固溶和时效处理方法来改善Ti6Al4V合金的微观结构和力学性能,并找到最佳的热处理工艺制度。

\section*{2. 试验}
试验材料是一种高强度的热轧Ti6Al4V钛合金,其厚度为6毫米。其化学成分为铁(Fe)<0.25\%、碳(C)<0.06\%、氢(H)<0.008\%、氮(N)<0.040\%、氧(O)<0.065\%、铝(Al)5.5$\sim$6.2\%、钒(V)3.5$\sim$4.0\%,其余为Ti。实验中选择了六个初步的热处理系统。它是根据相变的温度决定的。热处理过程如表1所示。为了确定($\alpha$+$\beta$)相形态对其断裂韧性的影响,Ti6Al4V合金在断裂韧性试验前进行了热处理。热处理是在SX-4-13型箱式电阻炉中进行的。热处理结束后,进行了微观结构分析、拉伸力学试验、仪器冲击试验、X射线衍射(XRD)试验和环境扫描电子显微镜(ESEM)对断裂形态的观察。热处理前后的Ti6Al4V合金管被加工成板条拉伸试样。在MTS810液压伺服万能试验机上进行了静态拉伸试验。拉伸试样包括的参数尺寸描述如下。厚度为7毫米。宽度是(20±0.05)毫米。测量长度为(60+0.5)毫米。夹持端长度为50毫米。平行段和夹持端之间的曲率长度大于或等于12。拉伸样品的总长度大于或等于184毫米。拉伸试验符合ISO 6892-1998的规定。测量了拉伸强度$\sigma_b$和屈服强度$\sigma_{0.2}$以及伸长率δ。断裂韧性是根据金属夏比缺口冲击试验法的标准,通过仪器冲击试验进行测试的。试样的尺寸为10mm×5mm×55mm。设备为ZBC2302-D型冲击试验机。其冲击能量为294J,冲击速度为5.24m/s。热处理前后的试样的微观结构用奥林巴斯PMG-3型显微镜进行了分析。试样被打磨成圆形,并在270K的试剂中进行蚀刻,该试剂为1mL HF+30mL HNO3+30mL H2O2。用FEI Quanta 450 ESEM观察冲击断裂。断口观察对于分析断裂特征和机制是必不可少的。
\section*{3. 分析与讨论}
$\alpha$相是($\alpha$+$\beta$)钛合金的基体相。$\alpha$相的数量、形状和大小直接决定了($\alpha$+$\beta$)钛合金的性能。在两相区,($\alpha$+$\beta$)相是通过不同的保温时间和温度的热处理得到的。该温度低于相变温度。($\alpha$+$\beta$)相的主要特征是晶粒形状不规则,晶界上有连续和不连续的$\alpha$相,以及许多小的次级$\alpha$相。颗粒内存在点状、球状、片状和短杆状的$\alpha$相。然而,当加热温度高于相变温度时,所有的($\alpha$+$\beta$)相都会转化为$\beta$相。晶粒的大小和形状不尽相同。它们是四边形、五边形和六边形。溶解和老化可以消除或减少连续晶界的$\alpha$相。它们可以显著提高拉伸和疲劳强度。但塑性会降低一些。溶解和时效处理可以明显改善疲劳强度。合金的$\beta$相越稳定,淬火后的$\beta$相就越容易转移,时效强化的效果会更好。当$\beta$稳定元素的温度达到$C_K$值时,会得到最大的效果。强化效果会随着$\beta$相的增加而降低。这导致了时效性$\beta$相的析出,以及$\alpha$相的数量下降。Ti6Al4V合金是($\alpha$+$\beta$)相合金。通过固溶和时效热处理可以改善其微观结构和力学性能,进而获得更好的综合性能。

图1中可以看到不同热处理后的Ti6Al4V合金的显微组织。根据得到的显微组织,加热温度在($\alpha$+$\beta$)→$\beta$跨度温度下,可以得到很多等轴结构。但移位组织的比例较少。当加热温度高于($\alpha$+$\beta$)→$\beta$ 转变温度时,可以得到粗晶和片状微结构。可以清楚地看到,原始的$\beta$晶粒和明显的$\alpha$相沿晶界分别出现。原有的$\beta$晶粒转变为长而交错的微观结构,在不同的地方编织起来。图1a是退火合金的微观结构。它是原生$\alpha$相和($\alpha$+$\beta$)相的混合物。从图1b到图1f可以看出,当合金经过溶液和时效处理时,其微观结构由$\beta$相和($\alpha$+$\beta$)相组成。但是时效后的微观结构更加粗大。图1b的淬火温度为920℃,淬火后$\alpha^\prime$相更少更小。$\alpha^\prime$相在时效处理后转化为精细和片状($\alpha$+$\beta$)相的混合物。$\alpha^\prime$相的尺寸随着老化温度的升高而变大。大尺寸的$\alpha^\prime$相在热处理后将转化为大尺寸的($\alpha$+$\beta$)相,具有较大的片状间距。从图1c可以看出。这是一个典型的两态微结构。当温度低于横截面温度时,可以得到它。与固溶体的$\alpha$相相比,老化后的$\alpha^\prime$相的尺寸明显变粗了。可以推测,从$\alpha^\prime$相析出的$\alpha$相不仅形成片状,而且还沿原$\alpha$相生长。因此,$\alpha$相的尺寸最终变得更粗。图1d显示了强化相的完全溶解度和合金元素在晶界上的均匀分布,随着溶液温度的上升。随着时效温度的上升,($\alpha$+$\beta$)相晶粒逐渐增加并变大。同时,$\beta$相重新结晶。由于$\beta$相的增加,在相变过程中,原子的扩散、相的溶解、沉淀和聚集,导致$\beta$相分布在$\alpha$晶粒附近的小区内。当温度接近于$\beta$晶体温度时,$\beta$相成为基体。这种微观结构具有良好的塑性和稳定性,但蠕变性能较差。马氏体在淬火过程中转化为$\alpha^\prime$相和变质的$\beta$相。从图1e可以看出,初级$\alpha$相完全转化为$\beta$相。片状的$\beta$相和存活的$\alpha$相呈组束状排列。$\alpha$相不仅沿晶界均匀分布,而且以束状的形式平行排列,嵌入$\beta$晶粒中。因此,得到了一个明显的篮子状的微观结构。结晶晶粒变小,所以综合性能得到改善。由于是水冷却,快速冷却过程中高温段的$\beta$相来不及转化为$\alpha$相。得到了马氏体$\alpha^{\prime\prime}$和变质态的$\beta$相。$\alpha^{\prime\prime}$和变质态$\beta$相开始分解,产生分散的($\alpha$+$\beta$)相篮状微观结构,具有良好的疲劳性能和其他综合性能。分裂的$\beta$相的尺寸越来越小,相互交错,微观结构细化。这改善了合金的综合性能。从图1e可以看出,随着溶液温度的升高,晶粒变粗,尺寸变大。此外,它是片状微观结构的形状,并出现明显的$\alpha$相。分层排列的$\alpha$相被嵌入$\beta$相中。一些残留的$\alpha$相沿着晶粒交错的微观结构不均匀地分布,像长条状。从图1g来看,像片状的晶粒是粗大的。原来的$\beta$晶粒转化为交错的微结构,如长条形。

表2显示了固溶和时效处理对Ti6Al4V合金的机械性能的影响。当热轧状态的合金经历了固溶和时效处理后,Ti6Al4V合金的强度得到了很大的提高。除个别工艺外,伸长率有一定程度的提高。5号工艺在960℃/1h+WQ和500℃/4h+AC的参数下可以获得最佳的综合性能。与热轧相比,屈服强度($\sigma_{0.2}$)提高50\%,极限拉伸强度($\sigma_b$)提高42\%,伸长率δ提高11\%。强度和伸长率随着淬火温度的升高而增加。但它们先是增加,然后减少。可以看出,最佳固溶温度为960℃,最佳时效温度为500℃。当Ti6Al4V合金在500℃的时效温度下进行热处理时,就像图1e一样得到网篮状的微观结构$\beta$相和($\alpha$+$\beta$)相混合物。它们分布在$\alpha$晶粒内的结晶边界附近。这将使合金具有良好的强度和伸长率。我们注意到,当固溶$\beta$相转化为马氏体($\alpha^\prime$相)时,Ti6Al4V合金的强度会增加。然后马氏体($\alpha^\prime$相)转化为细小的$\alpha$相和$\beta$相。根据上述描述,$\alpha$相减少,$\beta$相数量增加。随着淬火温度的逐渐升高,更多的$\alpha^\prime$相从$\beta$相转化而来。显然,$\alpha^\prime$相越多,得到的强度就越高。当溶液温度过高时,过多的粗大的$\alpha$相被保留下来。这导致了材料微观结构的不均匀,因此,由于应力集中,强度可能会下降。



表3显示了样品的冲击韧性。该结果与热轧样品的冲击韧性进行了比较。可以看出,冲击韧性值随着溶液温度的升高而降低。时效温度为450℃的冲击韧性比500℃的冲击韧性更稳定。当溶液温度为960℃时,获得最佳的冲击韧性,因为当淬火温度较高时,$\alpha$相的微观结构更粗。这将降低塑性和韧性。当溶液温度为920℃时,材料的韧性随着老化温度的上升而下降。当时效温度为450℃时,韧性$A_k$为40$J·cm^{-2}$;当时效温度为500℃时,韧性$A_k$为38.21$J·cm^{-2}$。随着溶液温度的升高,在一定的老化温度条件下,它们的相关曲线如图3所示。当溶液温度为1000℃时,韧度随老化温度的降低而增加。当老化温度为500℃时,韧性$A_k$为30.63$J·cm^{-2}$;当老化温度为450℃时,韧性$A_k$为33.05$J·cm^{-2}$。

为了分析热处理后材料的塑性断裂特征和微观结构之间的关系,对第3到第6道工序的冲击断裂进行了ESEM分析,以确定不同条件下材料的塑性。结果显示在图2中。图2a中的韧窝又深又大。它的塑性和韧性都很好。可以清楚地看到,断裂是沿着晶粒形成的。从图2b可以看出韧性较差的断裂特征。图2c中可以看到沿片状结构相的方向的层状断裂。厚片的层状微结构在材料上抵抗疲劳开裂的能力越来越低。图2d中的凹痕又深又大,分布均匀。这是明显的延性断裂。$\beta$变质相在溶液和老化处理后分解。$\alpha$相优先析出,并均匀地分布在晶界和$\beta$相中。最后($\alpha$+$\beta$)相结合,明显改善了强度和韧性,提高了综合性能。图2e的韧窝较浅,导致塑性和韧性差。


随着溶液温度的升高和老化温度的降低,大量的$\beta$转移相形成,$\alpha$相在晶界和领土上分布不均匀。在图2f中,断裂特征下降到准空隙断裂。值得注意的是,该断裂容易发生脆性断裂。在两相钛合金的基础上,Ti6Al4V钛合金具有小而均匀的球状和篮状混合微结构$\beta$相和($\alpha$+$\beta$)相。在仪器冲击断裂实验过程中,在原相和对话微观结构的边界会形成孔。随着冲击变形程度的增加,在$\beta$相跨群之前,这些孔沿着相的边界越来越大。混合相的微观结构与孔洞相比有所增长。裂纹的延伸具有阻挡作用。因此,机械性能受到其形状、分布、尺寸等方面的影响。结论是,两相的两态微观结构能有效地阻止空洞的增长和裂纹的扩展。

	原文著录:Liu Wanying, Lin Yuanhua, Chen Yuhai, et al. Effect of different heat treatments on microstructure and mechanical properties of ti6al4v titanium alloy  [J]. Rare Metal Materials and Engineering, 2017(634-639).
